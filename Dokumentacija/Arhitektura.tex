\chapter{Arhitektura i dizajn sustava}
		
		
		\textbf{\textit{dio 1. revizije}}\\

		\textit{ Potrebno je opisati stil arhitekture te identificirati: podsustave, preslikavanje na radnu platformu, spremišta podataka, mrežne protokole, globalni upravljački tok i sklopovsko-programske zahtjeve. Po točkama razraditi i popratiti odgovarajućim skicama:}
	\begin{itemize}
		\item 	\textit{izbor arhitekture temeljem principa oblikovanja pokazanih na predavanjima (objasniti zašto ste baš odabrali takvu arhitekturu)}
		\item 	\textit{organizaciju sustava s najviše razine apstrakcije (npr. klijent-poslužitelj, baza podataka, datotečni sustav, grafičko sučelje)}
		\item 	\textit{organizaciju aplikacije (npr. slojevi frontend i backend, MVC arhitektura) }		
	\end{itemize}

				
		\section{Baza podataka}
			
			\textbf{\textit{dio 1. revizije}}\\
			
		\textit{Potrebno je opisati koju vrstu i implementaciju baze podataka ste odabrali, glavne komponente od kojih se sastoji i slično.}
\textnormal{Za rješavanje projektnog zadatka odabrana je relacijska baza podataka. Implementirali smo je korištenjem sustava PostgreSQL. To je besplatan sustav za upravljanje bazom podataka otvorenog koda. Objekti relacijske baze podataka su relacije. Neformalno, i u PostgreSQL-u, to su dvodimenzionalne imenovane tablice gdje imenovani stupac predstavlja atribut, a redak zapis relacije. Izbor ovakve baze podataka omogućio nam je lakše strukturiranje podatka i definiranje veza između njih, normalizaciju, skalabilnost te sigurnost kod pohrane i dohvata podataka. 
Zbog potreba naše aplikacije, baza podataka sadrži entitete:}
             \begin{itemize}
		\item 	\textnormal{Korisnik}
		\item 	\textnormal{Djelatnik}
		\item 	\textnormal{Uloga}
                   \item 	\textnormal{Pacijent}
                   \item 	\textnormal{Terapija}
                   \item 	\textnormal{VrstaTerapije}	
                   \item 	\textnormal{Termin}			
                   \item 	\textnormal{Status}	
                   \item 	\textnormal{Oprema}
                   \item 	\textnormal{ZauzetaOprema}	
                   \item 	\textnormal{Soba}	
                   \item 	\textnormal{Uredaj}
                   \item 	\textnormal{VrstaUredaja}
                   				
	 \end{itemize}

		
			\subsection{Opis tablica}
			
 \textbf{Korisnik:}

 \textnormal{Entitet Korisnik sadrži sve bitne informacije o korisniku aplikacije. Entitet sadrži atribute: idKorisnika, ime, prezime, datumRodenja, adresa, email, lozinka. Korisnik može biti ili djelatnik zdravstvene ustanove ili pacijent koji se želi naručiti na terapiju u zdravstvenoj ustanovi.}

				
				
				\begin{longtblr}[
					label=none,
					entry=none
					]{
						width = \textwidth,
						colspec={|X[6,l]|X[6, l]|X[20, l]|}, 
						rowhead = 1,
					} %definicija širine tablice, širine stupaca, poravnanje i broja redaka naslova tablice
					\hline \SetCell[c=3]{c}{\textbf{Korisnik}}	 \\ \hline[3pt]
					\SetCell{LightGreen}idKorisnika & INT & jedinstveni identifikator korisnika 	\\ \hline
					ime & VARCHAR(50) & ime korisnika	\\ \hline 
                                               prezime & VARCHAR(50) & prezime korisnika	\\ \hline
                                               datumRodenja & DATE & datum rođenja korisnika	\\ \hline  
                                               adresa & VARCHAR(100) & adresa mjesta stanovanja korisnika	\\ \hline 
					 email & VARCHAR(50) & e-mail adresa korisnika   \\ \hline 
					 lozinka & VARCHAR(50) & hash lozinke za prijavu u aplikaciju	\\ \hline 
					 
				\end{longtblr}

\textbf{Djelatnik:}

\textnormal{Entitet Djelatnik je ekskluzivna specijalizacija entiteta Korisnik. Entitet sadrži atribute: idKorisnika, aktivan, idUloge.}

				\begin{longtblr}[
					label=none,
					entry=none
					]{
						width = \textwidth,
						colspec={|X[6,l]|X[6, l]|X[20, l]|}, 
						rowhead = 1,
					} %definicija širine tablice, širine stupaca, poravnanje i broja redaka naslova tablice
					\hline \SetCell[c=3]{c}{\textbf{Djelatnik}}	 \\ \hline[3pt]
					\SetCell{LightBlue}idKorisnika & INT & jedinstveni identifikator korisnika (korisnik.idKorisnika) 	\\ \hline
					aktivan & CHAR(2) & označava radni odnos liječnika i zdravstvene ustanove	\\ \hline 
					\SetCell{LightBlue}idUloge & INT & jedinstveni identifikator uloge (uloga.idUloge)	\\ \hline 
					 
				\end{longtblr}

\textbf{Uloga:}

\textnormal{Entitet Uloga služi za razlikovanje djelatnika zdravstvene ustanove. U aplikaciji razlikujemo liječnike od administratora. Entitet sadrži atribute: idUloge, imeUloge.}

				\begin{longtblr}[
					label=none,
					entry=none
					]{
						width = \textwidth,
						colspec={|X[6,l]|X[6, l]|X[20, l]|}, 
						rowhead = 1,
					} %definicija širine tablice, širine stupaca, poravnanje i broja redaka naslova tablice
					\hline \SetCell[c=3]{c}{\textbf{Uloga}}	 \\ \hline[3pt]
					\SetCell{LightGreen}idUloge & INT & jedinstveni identifikator uloge 	\\ \hline
					imeUloge & VARCHAR(50) & naziv uloge (liječnik ili administrator)	\\ \hline 
					 
				\end{longtblr}

\textbf{Pacijent:}

\textnormal{Entitet Pacijent je ekskluzivna specijalizacija entiteta Korisnik. Entitet sadrži atribute: idKorisnika, MBO.}

				\begin{longtblr}[
					label=none,
					entry=none
					]{
						width = \textwidth,
						colspec={|X[6,l]|X[6, l]|X[20, l]|}, 
						rowhead = 1,
					} %definicija širine tablice, širine stupaca, poravnanje i broja redaka naslova tablice
					\hline \SetCell[c=3]{c}{\textbf{Pacijent}}	 \\ \hline[3pt]
					\SetCell{LightBlue}idKorisnika & INT & jedinstveni identifikator korisnika (korisnik.idKorisnika)	\\ \hline
					MBO & CHAR(9) & matični broj osiguranika (pacijenta)	\\ \hline 

					 
				\end{longtblr}

\textbf{Terapija:}

\textnormal{Entitet Terapija sadrži sve bitne informacije o terapiji na koju je pacijent naručen. Entitet sadrži atribute: idTerapije, idLijecnika, opisOboljenja, zahtPostLijec, datumPoc, datumZavrs, idPacijenta, idDjelatnika, idVrste.}

				\begin{longtblr}[
					label=none,
					entry=none
					]{
						width = \textwidth,
						colspec={|X[6,l]|X[6, l]|X[20, l]|}, 
						rowhead = 1,
					} %definicija širine tablice, širine stupaca, poravnanje i broja redaka naslova tablice
					\hline \SetCell[c=3]{c}{\textbf{Terapija}}	 \\ \hline[3pt]
					\SetCell{LightGreen}idTerapije & INT & jedinstveni identifikator terapije	\\ \hline
					idLijecnika & INT & jedinstveni identifikator liječnika	\\ \hline 
                                               opisOboljenja & VARCHAR(200) & opis oboljenja pacijenta	\\ \hline
                                               zahtPostLijec & VARCHAR(200) & zahtjev za postupkom liječenja (terapijom)	\\ \hline  
                                               datumPoc & DATE & datum početka terapije	\\ \hline 
					 datumZavrs & DATE & datum završetka terapije   \\ \hline 
					 \SetCell{LightBlue}idPacijenta & INT & jedinstveni identifikator pacijenta (korisnik.idKorisnika)	\\ \hline 
					 \SetCell{LightBlue}idDjelatnika & INT & jedinstveni identifikator djelatnika (korisnik.idKorisnika)
					 \SetCell{LightBlue}idVrste & INT & jedinstveni identifikator vrste uređaja (vrstaTerapije.idVrste)
					 
				\end{longtblr}

\textbf{VrstaTerapije:}

\textnormal{Entitet VrstaTerapije sadrži sve bitne informacije o vrsti terapije na koju je pacijent naručen. Entitet sadrži atribute: idVrste, imeVrste, opisVrste.}

\begin{longtblr}[
					label=none,
					entry=none
					]{
						width = \textwidth,
						colspec={|X[6,l]|X[6, l]|X[20, l]|}, 
						rowhead = 1,
					} %definicija širine tablice, širine stupaca, poravnanje i broja redaka naslova tablice
					\hline \SetCell[c=3]{c}{\textbf{VrstaTerapije}}	 \\ \hline[3pt]
					\SetCell{LightGreen}idVrste & INT & jedinstveni identifikator vrste terapije	\\ \hline
					imeVrste & VARCHAR(50) & naziv vrste terapije	\\ \hline 
                                               opisVrste & VARCHAR(200) & opis vrste terapije	\\ \hline
					 
				\end{longtblr}

\textbf{Termin:}

\textnormal{Entitet Termin sadrži sve bitne informacije o terminu terapije na koji je pacijent naručen. Entitet Termin ne postoji bez entiteta vlasnika, entiteta Terapija. (Termin je egzistencijalno slab entitet.)}

\begin{longtblr}[
					label=none,
					entry=none
					]{
						width = \textwidth,
						colspec={|X[6,l]|X[6, l]|X[20, l]|}, 
						rowhead = 1,
					} %definicija širine tablice, širine stupaca, poravnanje i broja redaka naslova tablice
					\hline \SetCell[c=3]{c}{\textbf{Termin}}	 \\ \hline[3pt]
					\SetCell{LightGreen}idTermina & INT & jedinstveni identifikator termina terapije \\ \hline
					\SetCell{LightBlue}idTerapije & INT & jedinstveni identifikator terapije (terapija.idTerapije)	\\ \hline 
                                               od & DATETIME & datum i vrijeme početka termina	\\ \hline
					 do & DATETIME & datum i vrijeme završetka termina      \\ \hline
                                               komentar & VARCHAR(200) & komentar liječnika o napretku terapije \\ \hline
                                               \SetCell{LightBlue}brSobe & VARCHAR(10) & jedinstveni identifikator sobe (soba.brSobe)	\\ \hline
                                               \SetCell{LightBlue}idStatus & INT & jedinstveni identifikator statusa (status.idStatus)	\\ \hline
                                               \SetCell{Blue}idKorisnika & INT & jedinstveni identifikator korisnika (korisnik.idKorisnika)	\\ \hline
				\end{longtblr}

\textbf{Soba:}

\textnormal{Entitet Soba sadrži sve bitne informacije o sobi u kojoj se provodi neka terapija. Entitet sadrži atribute: brSobe, kapacitet.}

\begin{longtblr}[
					label=none,
					entry=none
					]{
						width = \textwidth,
						colspec={|X[6,l]|X[6, l]|X[20, l]|}, 
						rowhead = 1,
					} %definicija širine tablice, širine stupaca, poravnanje i broja redaka naslova tablice
					\hline \SetCell[c=3]{c}{\textbf{Soba}}	 \\ \hline[3pt]
					\SetCell{LightGreen}brSobe & VRACHAR(10) & jedinstveni identifikator sobe \\ \hline
                                               kapacitet & INT & kapacitet sobe, broj pacijenata koji istovremeno mogu biti na terapiji u nekoj sobi	\\ \hline
				\end{longtblr}

\textbf{Uredaj:}

\textnormal{Entitet Uredaj sadrži sve bitne informacije o uređaju koji se koristi u nekoj terapiji. Entitet sadrži atribute: idUredaja, brSobe, idVrste.}

\begin{longtblr}[
					label=none,
					entry=none
					]{
						width = \textwidth,
						colspec={|X[6,l]|X[6, l]|X[20, l]|}, 
						rowhead = 1,
					} %definicija širine tablice, širine stupaca, poravnanje i broja redaka naslova tablice
					\hline \SetCell[c=3]{c}{\textbf{Uredaj}}	 \\ \hline[3pt]
					\SetCell{LightGreen}idUredaja & INT & jedinstveni identifikator uređaja \\ \hline
                                               \SetCell{LightBlue}brSobe & VARCHAR(10) & jedinstveni identifikator sobe (soba.brSobe) \\ \hline
                                               \SetCell{LightBlue}idVrste & INT & jedinstveni identifikator vrste uređaja (vrstaUredaja.idVrste)\\ \hline
				\end{longtblr}

\textbf{VrstaUredaja:}

\textnormal{Entitet VrstaUredaja sadrži sve bitne informacije o vrsti uređaja koja se primijenjuje u nekoj terapiji. Entitet sadrži atribute: idVrste, imeVrste, opisVrste.}

\begin{longtblr}[
					label=none,
					entry=none
					]{
						width = \textwidth,
						colspec={|X[6,l]|X[6, l]|X[20, l]|}, 
						rowhead = 1,
					} %definicija širine tablice, širine stupaca, poravnanje i broja redaka naslova tablice
					\hline \SetCell[c=3]{c}{\textbf{VrstaUredaja}}	 \\ \hline[3pt]
					\SetCell{LightGreen}idVrste & INT & jedinstveni identifikator vrste uređaja \\ \hline
                                               imeVrste & VARCHAR(50) & naziv vrste uređaja \\ \hline
                                               opisVrste & VARCHAR(200) & opis vrste uređaja \\ \hline
				\end{longtblr}

\textbf{Oprema:}

\textnormal{Entitet Oprema sadrži sve bitne informacije o opremi koja je na raspolaganju u zdravstvenoj ustanovi. Entitet sadrži atribute: idOpreme, imeOpreme, opisOpreme.}

\begin{longtblr}[
					label=none,
					entry=none
					]{
						width = \textwidth,
						colspec={|X[6,l]|X[6, l]|X[20, l]|}, 
						rowhead = 1,
					} %definicija širine tablice, širine stupaca, poravnanje i broja redaka naslova tablice
					\hline \SetCell[c=3]{c}{\textbf{Oprema}}	 \\ \hline[3pt]
					\SetCell{LightGreen}idOpreme & INT & jedinstveni identifikator opreme \\ \hline
                                               imeOpreme & VARCHAR(50) & naziv opreme (npr. štake)\\ \hline
                                               opisOpreme & VARCHAR(200) & opis namjene opreme \\ \hline
				\end{longtblr}

\textbr{ZauzetaOprema:}

\textnormal{Entitet ZauzetaOprema sadrži sve bitne informacije o alokaciji opreme po terminima. Entitet sadrži atribute: idTermina, idTerapije, idOpreme.}

\begin{longtblr}[
					label=none,
					entry=none
					]{
						width = \textwidth,
						colspec={|X[6,l]|X[6, l]|X[20, l]|}, 
						rowhead = 1,
					} %definicija širine tablice, širine stupaca, poravnanje i broja redaka naslova tablice
					\hline \SetCell[c=3]{c}{\textbf{ZauzetaOprema}}	 \\ \hline[3pt]
					\SetCell{LightBlue}idTermina & INT & jedinstveni identifikator termina (termin.idTermina) \\ \hline
                                               \SetCell{LightBlue}idTerapije & INT & jedinstveni identifikator terapije (terapija.idTerapije) \\ \hline
                                               \SetCell{LightBlue}idOpreme & INT & jedinstveni identifikator opreme (oprema.idOpreme)  \\ \hline
				\end{longtblr}

\textbr{Status:}

\textnormal{Entitet Status sadrži sve bitne informacije o statusu termina terapija koji je dodijeljen nekom pacijentu. Entitet sadrži atribute: idStatus, imeStatus.}

\begin{longtblr}[
					label=none,
					entry=none
					]{
						width = \textwidth,
						colspec={|X[6,l]|X[6, l]|X[20, l]|}, 
						rowhead = 1,
					} %definicija širine tablice, širine stupaca, poravnanje i broja redaka naslova tablice
					\hline \SetCell[c=3]{c}{\textbf{Status}}	 \\ \hline[3pt]
					\SetCell{LightGreen}idStatus & INT & jedinstveni identifikator uređaja \\ \hline
                                               imeStatus & VARCHAR(50) & naziv statusa (npr. u tijeku, završen...) \\ \hline
     
				\end{longtblr}	
			
				\textit{Svaku tablicu je potrebno opisati po zadanom predlošku. Lijevo se nalazi točno ime varijable u bazi podataka, u sredini se nalazi tip podataka, a desno se nalazi opis varijable. Svjetlozelenom bojom označite primarni ključ. Svjetlo plavom označite strani ključ}
				
			
			\subsection{Dijagram baze podataka}
				\textit{ U ovom potpoglavlju potrebno je umetnuti dijagram baze podataka. Primarni i strani ključevi moraju biti označeni, a tablice povezane. Bazu podataka je potrebno normalizirati. Podsjetite se kolegija "Baze podataka".}
			
			\eject
			
		\section{Dijagram razreda}
		
			\textit{Potrebno je priložiti dijagram razreda s pripadajućim opisom. Zbog preglednosti je moguće dijagram razlomiti na više njih, ali moraju biti grupirani prema sličnim razinama apstrakcije i srodnim funkcionalnostima.}\\
			
			\textbf{\textit{dio 1. revizije}}\\
			
			\textit{Prilikom prve predaje projekta, potrebno je priložiti potpuno razrađen dijagram razreda vezan uz \textbf{generičku funkcionalnost} sustava. Ostale funkcionalnosti trebaju biti idejno razrađene u dijagramu sa sljedećim komponentama: nazivi razreda, nazivi metoda i vrste pristupa metodama (npr. javni, zaštićeni), nazivi atributa razreda, veze i odnosi između razreda.}\\
			
			\textbf{\textit{dio 2. revizije}}\\			
			
			\textit{Prilikom druge predaje projekta dijagram razreda i opisi moraju odgovarati stvarnom stanju implementacije}
			
			
			
			\eject
		
		\section{Dijagram stanja}
			
			
			\textbf{\textit{dio 2. revizije}}\\
			
			\textit{Potrebno je priložiti dijagram stanja i opisati ga. Dovoljan je jedan dijagram stanja koji prikazuje \textbf{značajan dio funkcionalnosti} sustava. Na primjer, stanja korisničkog sučelja i tijek korištenja neke ključne funkcionalnosti jesu značajan dio sustava, a registracija i prijava nisu. }
			
			
			\eject 
		
		\section{Dijagram aktivnosti}
			
			\textbf{\textit{dio 2. revizije}}\\
			
			 \textit{Potrebno je priložiti dijagram aktivnosti s pripadajućim opisom. Dijagram aktivnosti treba prikazivati značajan dio sustava.}
			
			\eject
		\section{Dijagram komponenti}
		
			\textbf{\textit{dio 2. revizije}}\\
		
			 \textit{Potrebno je priložiti dijagram komponenti s pripadajućim opisom. Dijagram komponenti treba prikazivati strukturu cijele aplikacije.}