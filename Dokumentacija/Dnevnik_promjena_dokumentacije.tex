\chapter{Dnevnik promjena dokumentacije}
		
		\textbf{\textit{Kontinuirano osvježavanje}}\\
				
		
		\begin{longtblr}[
				label=none
			]{
				width = \textwidth, 
				colspec={|X[2]|X[10]|X[5]|X[3]|}, 
				rowhead = 1
			}
			\hline
			\textbf{Rev.}	& \textbf{Opis promjene/dodatka} & \textbf{Autori} & \textbf{Datum}\\[3pt] \hline
			0.1 & Dovršen opis projektnog zadatka & E.Badurina & 13.11.2023. 		\\[3pt] \hline 
			0.2	& Dovršen opis arhitekture & L.Lasović & 13.11.2023. 	\\[3pt] \hline 
			0.2.1 & Dodan dio obrazaca uporabe & L.Lasović, E.Badurina & 13.11.2023. \\[3pt] \hline
			0.3 & Dodani svi obrasci uporabe & L.Lasović, E.Badurina & 14.11.2023. \\[3pt] \hline
			0.3.1 & Dodani dijagrami obrazaca uporabe & A.Jakovčević & 14.11.2023. \\[3pt] \hline
			0.3.2 & Revizija i promjene na obrascima uporabe & A.Jakovčević, E.Badurina & 15.11.2023. \\[3pt] \hline
			0.4 & Dodani sekvencijski dijagrami & L.Lasović, E.Badurina & 15.11.2023. \\[3pt] \hline
			0.4.1 & Dodan opis baze & L.Lasović & 15.11.2023. \\[3pt] \hline
			0.4.1 & Ispravak sekvencijskih dijagrama i ostali zahtjevi & E.Badurina, L.Lasović & 16.11.2023. \\[3pt] \hline
			0.5 & Dijagrami razreda & L.Akmačić & 17.11.2023.\\[3pt] \hline
			\textbf{1.0} & Verzija samo s bitnim dijelovima za 1. ciklus & * & 17.11.2023. \\[3pt] \hline 
			1.0.1 & Ispravak grešaka iz prvog ciklusa & E.Badurina \newline 21.12.2023 & * \\[3pt] \hline 
			1.1 & * & * \newline * & * \\[3pt] \hline 
			1.2 & * & * & * \\[3pt] \hline  
			\textbf{2.0} & *  & * & * \\[3pt] \hline 
			&  &  & \\[3pt] \hline	
		\end{longtblr}
	
	
		\textit{Moraju postojati glavne revizije dokumenata 1.0 i 2.0 na kraju prvog i drugog ciklusa. Između tih revizija mogu postojati manje revizije već prema tome kako se dokument bude nadopunjavao. Očekuje se da nakon svake značajnije promjene (dodatka, izmjene, uklanjanja dijelova teksta i popratnih grafičkih sadržaja) dokumenta se to zabilježi kao revizija. Npr., revizije unutar prvog ciklusa će imati oznake 0.1, 0.2, …, 0.9, 0.10, 0.11.. sve do konačne revizije prvog ciklusa 1.0. U drugom ciklusu se nastavlja s revizijama 1.1, 1.2, itd.}