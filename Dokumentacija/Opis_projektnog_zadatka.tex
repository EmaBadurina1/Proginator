
\chapter{Opis projektnog zadatka}
		
		\textbf{\textit{dio 1. revizije}}\\
		
		
		Cilj ovog projekta je izgradnja web aplikacije koja će omogućiti ljudima/bolesnicima lakše prijavljivanje na slobodne termine za medicinsku rehabilitaciju i fizikalnu terapiju te praćenje njihovog zdravstvenog napretka. Također će omogućiti zaposlenicima ustanove da odbijaju ili prihvaćaju termine ovisno o raspoloživosti prostorija, opreme i osoblja pritom imajući mogućnost uvida u pacijentove prošle terapije i napredak. Vrijeme porovođenja rehabilitacije je svakim radnim danom od ponedjeljka do petka od 8:00 do 20:00 sati.
		
		\noindent Aplikacija će razlikovati tri vrste korisnika: 
		\begin{packed_item}
			
			\item  pacijenta/bolesnika
			\item  liječnika - djelatnika zdravstvene ustanove
			\item  administratora
		\end{packed_item}
		
		Prilikom pokretanja web aplikacije svaki korisnik unosi svoj e-mail i lozinku. Ovisno o vrsti korisnika biti će preusmjereni na različite stranice.
		
		\textit{Pacijent} se samostalno prijavljuje u sustav. U slučaju da još ne postoji imati će opciju registracije. Za registraciju mora unijeti: 
		\begin{packed_item}
			\item ime
			\item prezime
			\item e-mail adresu
			\item MBO - Matični Broj Osiguranika
			\item broj telefona
			\item lozinku
		\end{packed_item}
		Prilikom registracije pomoću MBO-a provjerava se ako korisnik postoji u središnjem informacijskom sustavu zdravstvene zaštite.
		Nakon prijave ili registracije korisnik je preusmjeren na stranicu gdje se prikazuju njegovi termini i zahtjevi. U terminima se prikazuju protekli i budući termini, za protekle termine piše ako je pacijent došao te komentari djelatnika ustanove o napretku. Termin se dobiva nakon što se odobri zahtjev za jednim.		
		Zahtjev se sastoji od:
		\begin{packed_item}
			\item vremena predaje
			\item željenog termina
			\item vrste/tipa rehabilitacije
			\item liječnika koji je uputio pacijenta na terapiju
			\item reference na prošlu terapiju
			\item statusa
		\end{packed_item}
		Status može biti odobreno, odbijeno, čeka na odobrenje ili isteklo u slučaju da nitko nije odobrio niti odbio termin do željenog vremena terapije.
		Klikom na opciju \textit{naruči se} pacijent može napraviti novi zahtjev za terminom. Zahtjev popunjava unoseći vrstu rehabilitacije, liječnika koji ga je uputio, ukoliko se radi o ponovljenoj terapiji odabrati će i termin već obavljenog postupka terapije u ustanovi, također u komentar za liječnika dodaje opis svojih oboljenja. Status liječnika koji je izdao uputnicu provjerava se u imeniku liječnika. U odabiru termina biti će prikazano prvih nekoliko slobodnih termina za odabrani tip rehabilitacije i opcije da se unese željeni datum i vrijeme. Nakon odabira željenog termina pacijent predaje zahtjev. Pacijent će o odobrenju ili odbijanju zahtjeva, kao i svim mogućim promjenama biti obaviješten mailom.
		
		\textit{Liječnik} nakon prijavljivanja u sustav ima pregled svih pacijenata i njihovih podataka, klikom na prikaz terapije bit će mu prikazani svi termini odabranog pacijenta i detalji o njima, tu će imati opciju evidentirati dolazak pacijenta i zabilježiti komentare vezane uz napredak ili pregledati napredak i komentare iz prošlih termina. Također svaki djelatnik ustanove imati će mogućnost prihvaćanja ili odbijanja termina ovisno o raspoloživosti prostorija i opreme. Liječnik koji prihvati zahtjev automatski će se dodati kao liječnik koji će tu terapiju izvoditi. 
		
		\textit{Administrator} ima pregled svih pacijenata i djelatnika. Uz ovlasti koje imaju djelatnici administrator pri zaposlenju novog liječnika izrađuje korisnički račun za njega s inicijalnom lozinkom. Također nakon prestanka radnog odnosa administrator može ukloniti tog liječnika. Administrator definira sve što je potrebno za ispravan rad sustava, dakle može mijenjati broj opreme, dostupne prostorije, .
		
		---
		
		Ovim projektom smanjio bi se opseg posla djelatnika ustanove 
		---
		
		Aplikacija će se moći proširiti da djelatnici uopće ne moraju prihvaćati ili odbijati prijave, već će to sustav raditi automatski na temelju raspoloživih podataka o opremi, prostorijama, djelatnicima i već zauzetim terminima. Svi termini, prostorije, oprema i djelatnici vidljivi su adminu koji im može mijenjati status za određeno vrijeme, na primjer kada liječnik ode na godišnji odmor može promijeniti njegov status u neaktivan za to razdoblje ili kada cijela ustanova ima sastanak onemogućiti sve termine za vrijeme tog sastanka. Ostala bi mogućnost upućivanja maila ili telefonskog poziva pacijentu u slučaju ikakvih promjena. 
		
		
		---
		
	
		
		\begin{packed_item}
			\item \textit{potencijalna korist ovog projekta}
			\item \textit{postojeća slična rješenja (istražiti i ukratko opisati razlike u odnosu na zadani zadatak). Dodajte slike koja predočavaju slična rješenja.}
			\item \textit{skup korisnika koji bi mogao biti zainteresiran za ostvareno rješenje.}
			\item \textit{mogućnost prilagodbe rješenja }
			\item \textit{opseg projektnog zadatka}
			\item \textit{moguće nadogradnje projektnog zadatka}
		\end{packed_item}
		
		\textit{Za pomoć pogledati reference navedene u poglavlju „Popis literature“, a po potrebi konzultirati sadržaj na internetu koji nudi dobre smjernice u tom pogledu.}
		\eject
		
		\section{Primjeri u \LaTeX u}
		
		\textit{Ovo potpoglavlje izbrisati.}\\

		U nastavku se nalaze različiti primjeri kako koristiti osnovne funkcionalnosti \LaTeX a koje su potrebne za izradu dokumentacije. Za dodatnu pomoć obratiti se asistentu na projektu ili potražiti upute na sljedećim web sjedištima:
		\begin{itemize}
			\item Upute za izradu diplomskog rada u \LaTeX u - \url{https://www.fer.unizg.hr/_download/repository/LaTeX-upute.pdf}
			\item \LaTeX\ projekt - \url{https://www.latex-project.org/help/}
			\item StackExchange za Tex - \url{https://tex.stackexchange.com/}\\
		
		\end{itemize} 	


		
		\noindent \underbar{podcrtani tekst}, \textbf{podebljani tekst}, 	\textit{nagnuti tekst}\\
		\noindent \normalsize primjer \large primjer \Large primjer \LARGE {primjer} \huge {primjer} \Huge primjer \normalsize
				
		\begin{packed_item}
			
			\item  primjer
			\item  primjer
			\item  primjer
			\item[] \begin{packed_enum}
				\item primjer
				\item[] \begin{packed_enum}
					\item[1.a] primjer
					\item[b] primjer
				\end{packed_enum}
				\item primjer
			\end{packed_enum}
			
		\end{packed_item}
		
		\noindent primjer url-a: \url{https://www.fer.unizg.hr/predmet/proinz/projekt}
		
		\noindent posebni znakovi: \# \$ \% \& \{ \} \_ 
		$|$ $<$ $>$ 
		\^{} 
		\~{} 
		$\backslash$ 
		
		
		\begin{longtblr}[
			label=none,
			entry=none
			]{
				width = \textwidth,
				colspec={|X[8,l]|X[8, l]|X[16, l]|}, 
				rowhead = 1,
			} %definicija širine tablice, širine stupaca, poravnanje i broja redaka naslova tablice
			\hline \SetCell[c=3]{c}{\textbf{naslov unutar tablice}}	 \\ \hline[3pt]
			\SetCell{LightGreen}IDKorisnik & INT	&  	Lorem ipsum dolor sit amet, consectetur adipiscing elit, sed do eiusmod  	\\ \hline
			korisnickoIme	& VARCHAR &   	\\ \hline 
			email & VARCHAR &   \\ \hline 
			ime & VARCHAR	&  		\\ \hline 
			\SetCell{LightBlue} primjer	& VARCHAR &   	\\ \hline 
		\end{longtblr}
		

		\begin{longtblr}[
				caption = {Naslov s referencom izvan tablice},
				entry = {Short Caption},
			]{
				width = \textwidth, 
				colspec = {|X[8,l]|X[8,l]|X[16,l]|}, 
				rowhead = 1,
			}
			\hline
			\SetCell{LightGreen}IDKorisnik & INT	&  	Lorem ipsum dolor sit amet, consectetur adipiscing elit, sed do eiusmod  	\\ \hline
			korisnickoIme	& VARCHAR &   	\\ \hline 
			email & VARCHAR &   \\ \hline 
			ime & VARCHAR	&  		\\ \hline 
			\SetCell{LightBlue} primjer	& VARCHAR &   	\\ \hline 
		\end{longtblr}
	


		
		
		%unos slike
		\begin{figure}[H]
			\includegraphics[scale=0.4]{slike/aktivnost.PNG} %veličina slike u odnosu na originalnu datoteku i pozicija slike
			\centering
			\caption{Primjer slike s potpisom}
			\label{fig:promjene}
		\end{figure}
		
		\begin{figure}[H]
			\includegraphics[width=\textwidth]{slike/aktivnost.PNG} %veličina u odnosu na širinu linije
			\caption{Primjer slike s potpisom 2}
			\label{fig:promjene2} %label mora biti drugaciji za svaku sliku
		\end{figure}
		
		Referenciranje slike \ref{fig:promjene2} u tekstu.
		
		\eject
		
	