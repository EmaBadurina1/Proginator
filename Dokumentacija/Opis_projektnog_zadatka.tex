
\chapter{Opis projektnog zadatka}
		
		Cilj ovog projekta je izgradnja web aplikacije koja će omogućiti ljudima/pacijentima lakše prijavljivanje na slobodne termine za medicinsku rehabilitaciju i fizikalnu terapiju te praćenje njihovog zdravstvenog napretka. Također će omogućiti zaposlenicima ustanove da odbijaju ili prihvaćaju termine ovisno o raspoloživosti prostorija, opreme i osoblja pritom imajući mogućnost uvida u pacjentove prošle terapije i napredak. Vrijeme provođenja rehabilitacije je svakim radnim danom od ponedjeljka do petka od 8:00 do 20:00 sati.
		
		\noindent Aplikacija će razlikovati tri vrste korisnika: 
		\begin{packed_item}
			
			\item  pacijenta
			\item  liječnika - djelatnika zdravstvene ustanove
			\item  administratora - djelatnika zdravstvene ustanove
		\end{packed_item}
		
		Prilikom pokretanja web aplikacije svaki korisnik unosi svoju e-mail adresu i lozinku. Ovisno o vrsti korisnika biti će preusmjereni na različite stranice. Svaki korisnik imati će mogućnost mijenjanja nekih osobnih podataka i lozinke. Kao i promjene lozinke u slučaju da je zaboravljena.
		
		\textit{Pacijent} se samostalno prijavljuje u sustav. U slučaju da još ne postoji imati će opciju registracije. Za registraciju mora unijeti: 
		\begin{packed_item}
			\item ime
			\item prezime
			\item e-mail adresu
			\item MBO - Matični Broj Osiguranika
			\item broj telefona
			\item lozinku
		\end{packed_item}
		Prilikom registracije pomoću MBO-a provjerava se ako korisnik postoji u središnjem informacijskom sustavu zdravstvene zaštite.
		Nakon prijave korisnik je preusmjeren na početnu stranicu gdje može prikazati svoje termine, terapije i prijavljivati iste. U terminima se prikazuju protekli i budući termini, za protekle termine piše ako je pacijent došao te komentari djelatnika ustanove o napretku. Termin se dobiva nakon što se odobri zahtjev za jednim. No prije nego pacijent može birati termin mora kreirati terapiju, u slučaju da nema ni jednu aktivnu. Terapiju kreira unoseći podatke:
		\begin{packed_item}
			\item ime i prezime liječnika koji ga je uputio na terapiju
			\item tip rehabilitacije
			\item opis oboljenja 
			\item opis postupka liječenja
		\end{packed_item}
		Nakon što preda podatke o terapiji, provjerava se status liječnika u imeniku liječnika. U slučaju da uneseni liječnik ne postoji, zbog krivo unesenih podataka sustav će upozoriti pacijenta o tome i imati će priliku ispraviti podatke. Nakon uspješno kreirane terapije pacijent može kreirati termin. 
		Termin se sastoji od:
		\begin{packed_item}
			\item željenog vremena 
			\item vrste/tipa rehabilitacije
			\item reference na terapiju
			\item komentara za liječnika
		\end{packed_item}
		Pacijent odabire željeni termin tako što prvo odabire referencu na neku od njegovih aktivnih terapija, nakon toga unosi datum, na temelju kojeg će mu se prikazati slobodni termini. Pacijent će odabrati jedan, opcionalno dodati komentar i poslati zahtjev. Ukoliko se željeni termin nije popunio, termin će automatski biti odobren. Pacijentu će o odobrenom terminu, te o svim mogućim promjenama biti obaviješten e-mailom.
		
		\textit{Liječnik} nakon prijavljivanja u sustav ima pregled svih pacijenata i njihovih podataka. Imati će opciju pretraživanja pacijenta. Klikom na prikaz terapije bit će mu prikazani svi termini odabranog pacijenta i detalji o njima, tu će imati opciju evidentirati dolazak pacijenta i zabilježiti komentare vezane uz napredak ili pregledati napredak i komentare iz prošlih termina. Liječnik koji je evidentirao pacijenta dodati će se u sustav kao liječnik koji je vodio taj termin terapije. Također svaki djelatnik ustanove imati će mogućnost promjene termina, o čemu će pacijent biti obaviješten mailom. 
		
		\textit{Administrator} ima pregled svih pacijenata i djelatnika. Uz ovlasti koje imaju liječnici, administrator pri zaposlenju novog liječnika izrađuje korisnički račun za njega s inicijalnom lozinkom. Također nakon prestanka radnog odnosa administrator može ukloniti tog liječnika. Administrator definira sve što je potrebno za ispravan rad sustava, dakle može mijenjati dostupnost opreme i unositi novu opremu, mijenjati dostupnost prostorija i unositi nove. Također ako dođe do promjena u dostupnosti prostorija ili opreme.
		
		---
		
		Ovim projektom smanjio bi se opseg posla djelatnika ustanove i olakšao proces prijave na rehabilitaciju pacijentima, umjesto prijavljivanja u živo i rješavanja papirologije pacijentima će biti omogućeno automatsko prijavljivanje na slobodne termine, a liječnicima evidentiranje istih. 
		
		---
		
		Odgovarajući primjer slične aplikacije na istu temu nije pronađen, moguće, zbog zatvorenosti ovakvog tipa aplikacije prema javnosti, ali primjer sa sličnim funkcijama iako ne u istu svrhu je booking.com (\ref{fig:booking}). Na početnoj stranici \textit{booking}-a nailazimo na opcije prijave i registracije, iako za početak korištenja same stranice to nije nužno kao kod ove aplikacije. Korisnik \textit{booking}-a rezervira apartman, dok korisnik aplikacije za rehabilitaciju rezervira svoj termin za terapiju. Kao što korisnik na \textit{booking}-u unosi grad u kojem želi rezervirati apartman, korisnik u ovoj aplikaciji odabire tip rehabilitacije za koju se želi prijaviti. U oba slučaja kod prijave na željeni termin ili rezervacije apartmana korisnik dobiva neke ponuđene termine/opcije ili može samostalno odabrati vrijeme željenog termina/rezervacije. 
		
		\begin{figure}[H]
			\includegraphics[scale=0.4]{slike/slicna_aplikacija.PNG} %veličina slike u odnosu na originalnu datoteku i pozicija slike
			\centering
			\caption{Primjer slične aplikacije}
			\label{fig:booking}
		\end{figure}
		
		---
		
		Aplikacija će se moći proširiti na način da su svi termini, prostorije, oprema i djelatnici vidljivi su administratoru koji im može mijenjati status za određeno vrijeme, na primjer kada liječnik ode na godišnji odmor može promijeniti njegov status u neaktivan za to razdoblje ili kada cijela ustanova ima sastanak onemogućiti sve termine za vrijeme tog sastanka. Ostala bi mogućnost upućivanja maila ili telefonskog poziva pacijentu u slučaju ikakvih promjena.
		
	
		\eject
		
				
	