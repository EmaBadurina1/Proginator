\chapter{Specifikacija programske potpore}
		
	\section{Funkcionalni zahtjevi}
			
			\textbf{\textit{dio 1. revizije}}\\
			
			\textit{Navesti \textbf{dionike} koji imaju \textbf{interes u ovom sustavu} ili  \textbf{su nositelji odgovornosti}. To su prije svega korisnici, ali i administratori sustava, naručitelji, razvojni tim.}\\
				
			\textit{Navesti \textbf{aktore} koji izravno \textbf{koriste} ili \textbf{komuniciraju sa sustavom}. Oni mogu imati inicijatorsku ulogu, tj. započinju određene procese u sustavu ili samo sudioničku ulogu, tj. obavljaju određeni posao. Za svakog aktora navesti funkcionalne zahtjeve koji se na njega odnose.}\\
			
			
			\noindent \textbf{Dionici:}
			
			\begin{packed_enum}
				
				\item Pacijent
				\item Djelatnik	
				\begin{packed_item}
					\item Liječnik
					\item Administrator
				\end{packed_item}
				\item Razvojni tim
				
			\end{packed_enum}
			
			Pacijenti i djelatnici zajedničkim imenom su korisnici.
			
			
			\noindent \textbf{Aktori i njihovi funkcionalni zahtjevi:}
			
			
			\begin{packed_enum}
				\item  \underbar{Neregistrirani korisnik (inicijator) može:}
				\begin{packed_enum}
					\item registrirati se u sustav, unijeti potrebne podatke(ime, prezime, e-mail adresu, MBO, broj telefona, lozinku)
				\end{packed_enum}
			
				\item  \underbar{Neprijavljeni korisnik (inicijator) može:}
				\begin{packed_enum}
					\item prijaviti se u sustav koristeći svoju e-mail adresu i lozinku
				\end{packed_enum}
				
				\item  \underbar{Pacijent-\textit{prijavljeni korisnik} (inicijator) može:}
				\begin{packed_enum}
					\item vidjeti svoje zahtjeve za terminom terapije (vrijeme predaje zahtjeva, vrijeme željenog termina, tip rehabilitacije, liječnika koji ga je uputio na rehabilitaciju, referencu na prošlu terapiju i status) 
					\item vidjeti svoje termine (vrijeme termina, prostoriju, tip rehabilitacije, liječnika, napomene i ishod termina) 
					\item filtrirati svoje termine
					\item prikazati nalaz (detalji i komentari) s termina odrađene terapije
					\item naručiti se na terapiju, unijeti tip rehabilitacije, liječnika koji ga je uputio na terapiju i opcionalno referencu na prošlu terapiju; također može odabrati predložene termine ili unijeti nepredloženi termin i dodati napomenu za liječnika
					\item ...
				\end{packed_enum}
				
				\item  \underbar{Djelatnik-\textit{prijavljeni korisnik} (inicijator) može:}
				\begin{packed_enum}
					
					\item vidjeti popis svih pacijenata (ime, prezime, MBO, e-mail adresa, broj telefona)
					\item prikazati termine pojedinog pacijenta i prikazati evidenciju termina pojedinog pacijenta
					\item evidentirati dolazak pacijenta na termin, dodati komentar o odrađenom terminu i vidjeti informacije o terapiji
					\item prihvatiti/odbiti zahtjev za terminom na temelju raspoloživih prostorija i opreme 
					
				\end{packed_enum}
				\item  \underbar{Administrator - \textit{prijavljeni korisnik} (inicijator)  može:}
				\begin{packed_enum}
					
					\item vidjeti popis svih djelatnika (ime, prezime, OIB, e-mail adresa)
					\item urediti podatke djelatnika
					\item ukloniti djelatnika
					\item dodati/registrirati novog djelatnika (ime, prezime, e-mail adresa, OIB, lozinka)
					\item mijenjati raspoloživost soba i opreme
					\item dodavati opremu
					\item ????
					
				\end{packed_enum}
				
				\item  \underbar{Baza podataka (sudionik) može:}
				\begin{packed_enum}
					
					\item pohranjuje podatke i ovlasti korisnika
					\item pohranjuje podatke o opremi i prostorijama
					\item pohranjuje podatke o terapijama i terminima
					
				\end{packed_enum}
			\end{packed_enum}
			
			\eject 
			
			
				
			\subsection{Obrasci uporabe}
				
				\textbf{\textit{dio 1. revizije}}
				
				\subsubsection{Opis obrazaca uporabe}
					\textit{Funkcionalne zahtjeve razraditi u obliku obrazaca uporabe. Svaki obrazac je potrebno razraditi prema donjem predlošku. Ukoliko u nekom koraku može doći do odstupanja, potrebno je to odstupanje opisati i po mogućnosti ponuditi rješenje kojim bi se tijek obrasca vratio na osnovni tijek.}\\
					

					\noindent \underbar{\textbf{UC1 - Prijava u sustav}}
					\begin{packed_item}
	
						\item \textbf{Glavni sudionik: }Korisnik
						\item  \textbf{Cilj: } Prijaviti se u sustav
						\item  \textbf{Sudionici: } Baza podataka
						\item  \textbf{Preduvjet: } Korisnik je registriran u sustav
						\item  \textbf{Opis osnovnog tijeka: }
						
						\item[] \begin{packed_enum}
	
							\item Korisnik unosi svoju e-mail adresu i lozinku
							\item Provjera ispravnosti podataka
							\item Prikaz početne stranice ovisno o korisniku
						\end{packed_enum}
						
						\item  \textbf{Opis mogućih odstupanja:}
						
						\item[] \begin{packed_item}
	
							\item[2.a] Neispravna e-mail adresa ili lozinka
							\item[] \begin{packed_enum}
								
								\item obavijest o neispravnosti unesenih podataka
								
							\end{packed_enum}
							
						\end{packed_item}
					\end{packed_item}
				
				\noindent \underbar{\textbf{UC2 - Registracija}}
				\begin{packed_item}
					
					\item \textbf{Glavni sudionik: }Neregistrirani korisnik
					\item  \textbf{Cilj: }Registrirati se u sustav
					\item  \textbf{Sudionici: }Baza podataka
					\item  \textbf{Preduvjet: } -  
					\item  \textbf{Opis osnovnog tijeka: }
					
					\item[] \begin{packed_enum}
						
						\item Korisnik unosi svoje podatke (ime, prezime, e-mail adresu, MBO, lozinka)
						\item Provjera ispravnosti podataka verifikacijom iz baze podataka sustava zdravstvene zaštite
						\item prikaz početne stranice
					\end{packed_enum}
					
					\item  \textbf{Opis mogućih odstupanja:}
					
					\item[] \begin{packed_item}
						
						\item[2.a] Korisnik ne postoji u sustavu zdravstvene zaštite
						\item[] \begin{packed_enum}
							
							\item Sustav obavještava korisnika o neispravnosti podataka
							\item Korisnik mijenja podatke ili odustaje od registracije
							
						\end{packed_enum}
						
					\end{packed_item}
				\end{packed_item}
				
				\noindent \underbar{\textbf{UC3 - pregled osobnih podataka}}
				\begin{packed_item}
					
					\item \textbf{Glavni sudionik: }Korisnik
					\item  \textbf{Cilj: }Pregledati osobne podatke 
					\item  \textbf{Sudionici: }Baza podataka
					\item  \textbf{Preduvjet: }Korisnik mora biti prijavljen u sustav
					\item  \textbf{Opis osnovnog tijeka: }
					
					\item[] \begin{packed_enum}
						
						\item Korisnik odabire opciju "Profil"
						\item Sustav prikazuje korisnikove osobne podatke 
					\end{packed_enum}
					
				\end{packed_item}
				\noindent \underbar{\textbf{UC4 - Promjena osobnih podataka}}
				\begin{packed_item}
					
					\item \textbf{Glavni sudionik: }Korisnik
					\item  \textbf{Cilj: }Promijeniti osobne podatke
					\item  \textbf{Sudionici: }Baza podataka
					\item  \textbf{Preduvjet: }Korisnik mora biti prijavljen u sustav
					\item  \textbf{Opis osnovnog tijeka: }
					
					\item[] \begin{packed_enum}
						
						\item Korisnik odabire opciju "Uredi"  
						\item Korisnik unosi/mijenja podatke
						\item Korisnik odabire opciju "Spremi promjene"
					\end{packed_enum}
				\end{packed_item}
				
				\noindent \underbar{\textbf{UC5 - Promjena lozinke}}
				\begin{packed_item}
					
					\item \textbf{Glavni sudionik: }Korisnik
					\item  \textbf{Cilj: }Promijeniti lozinku
					\item  \textbf{Sudionici: }Baza podataka
					\item  \textbf{Preduvjet: }Korisnik je prijavljen u sustav
					\item  \textbf{Opis osnovnog tijeka: }
					
					\item[] \begin{packed_enum}
						
						\item Korisnik odabire opciju "Promijeni lozinku"
						\item Korisnik potvrđuje staru lozinku
						\item Korisnik unosi novu lozinku
						\item Korisnik potvrđuje novu lozinku
						\item Sustav korisnika prebacuje na prijavu
						\item Korisnik se prijavljuje s novom lozinkom
					\end{packed_enum}
					
					\item  \textbf{Opis mogućih odstupanja:}
					
					\item[] \begin{packed_item}
						
						\item[2.a] korisnik unosi neispravnu staru lozinku
						\item[] \begin{packed_enum}
							
							\item Sustav upozorava korisnika o neispravnosti lozinke
							\item Korisnik ponovno unosi lozinku ili odustaje od promjene
							
						\end{packed_enum}
						\item[3.a] Nova lozinka je jednaka staroj
						\item[]\begin{packed_enum}
							
							\item Sustav upozorava korisnika da su mu lozinke iste
							\item Korisnik mijenja novu lozinku ili odustaje od promjene
							
						\end{packed_enum}
						
					\end{packed_item}
				\end{packed_item}
				
				\noindent \underbar{\textbf{UC6 - Kreiranje terapije}}
				\begin{packed_item}
					
					\item \textbf{Glavni sudionik: }Pacjient
					\item  \textbf{Cilj: }Kreiranje procesa terapije
					\item  \textbf{Sudionici: }Baza podataka
					\item  \textbf{Preduvjet: }Pacijent je prijavljen u sustav
					\item  \textbf{Opis osnovnog tijeka: }
					
					\item[] \begin{packed_enum}
						
						\item Pacijent unosi ime i prezime liječnika, opis oboljenja i zahtijevani postupak liječenja, te odabire vrstu terapije
						\item Pacijent odabire opciju "Pošalji"
						\item Provjera ispravnosti podataka u imeniku liječnika
						\item Prikaz stranice za odabir termina
					\end{packed_enum}
					
					\item  \textbf{Opis mogućih odstupanja:}
					
					\item[] \begin{packed_item}
						
						\item[3.a] Liječnik ne postoji u imeniku liječnika
						\item[] \begin{packed_enum}
							
							\item Sustav obavještava pacijenta o neispravnim podacima(ime i prezime liječnika)
							\item Pacijent mijenja podatke ili odustaje od kreiranja terapije
							
						\end{packed_enum}
												
					\end{packed_item}
				\end{packed_item}
				\noindent \underbar{\textbf{UC7 - Odabir termina}}
				\begin{packed_item}
					
					\item \textbf{Glavni sudionik: }Pacijent
					\item  \textbf{Cilj: }Poslati zahtjev za željenim terminom
					\item  \textbf{Sudionici: }Baza podataka
					\item  \textbf{Preduvjet: }Korisnik je prijavljen u sustav
					\item  \textbf{Opis osnovnog tijeka: }
					
					\item[] \begin{packed_enum}
						
						\item Pacijent odabire terapiju za koju se prijavljuje na termin
						\item Pacijent odabire jedan od ponuđenih termina
						\item Pacijent opcionalno dodaje napomenu za liječnika
						\item Pacijent šalje zahtjev klikom na "Pošalji" 
					\end{packed_enum}
					
					\item  \textbf{Opis mogućih odstupanja:}
					
					\item[] \begin{packed_item}
						
						\item[2.a] Pacijent 
						\item[] \begin{packed_enum}
							
							\item 
							
						\end{packed_enum}
						\item[3.a] Nova lozinka je jednaka staroj
						\item[]\begin{packed_enum}
							
							\item 
							
						\end{packed_enum}
						
					\end{packed_item}
				\end{packed_item}
					
				\subsubsection{Dijagrami obrazaca uporabe}
					
					\textit{Prikazati odnos aktora i obrazaca uporabe odgovarajućim UML dijagramom. Nije nužno nacrtati sve na jednom dijagramu. Modelirati po razinama apstrakcije i skupovima srodnih funkcionalnosti.}
				\eject		
				
			\subsection{Sekvencijski dijagrami}
				
				\textbf{\textit{dio 1. revizije}}\\
				
				\textit{Nacrtati sekvencijske dijagrame koji modeliraju najvažnije dijelove sustava (max. 4 dijagrama). Ukoliko postoji nedoumica oko odabira, razjasniti s asistentom. Uz svaki dijagram napisati detaljni opis dijagrama.}
				\eject
	
		\section{Ostali zahtjevi}
		
			\textbf{\textit{dio 1. revizije}}\\
		 
			 \textit{Nefunkcionalni zahtjevi i zahtjevi domene primjene dopunjuju funkcionalne zahtjeve. Oni opisuju \textbf{kako se sustav treba ponašati} i koja \textbf{ograničenja} treba poštivati (performanse, korisničko iskustvo, pouzdanost, standardi kvalitete, sigurnost...). Primjeri takvih zahtjeva u Vašem projektu mogu biti: podržani jezici korisničkog sučelja, vrijeme odziva, najveći mogući podržani broj korisnika, podržane web/mobilne platforme, razina zaštite (protokoli komunikacije, kriptiranje...)... Svaki takav zahtjev potrebno je navesti u jednoj ili dvije rečenice.}
			 
			 
			 
	