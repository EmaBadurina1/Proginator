\chapter{Zaključak i budući rad}
		 
		 Zadatak grupe Proginator bio je izrada aplikacije za medicinsku rehabilitaciju koja omogućava pacijentima prijavljivanje na različite terapije i termine unutar njih, a djelatnicima pregled terapija i evidenciju termina. Nakon 16 tjedana rada u timu ostvarili smo zadani cilj. Projekt se provodio u dvije faze.
		 
		 U prvoj fazi projekta upoznali smo tim i raspravili o znanjima pojedinih članova na temelju čega smo djelomično podijelili zadatke. Također u prvoj fazi raspravili smo što želimo od aplikacije i o tome raspravili s asistentom na temelju čega smo pisali dokumentaciju, posebice obrasce uporabe koji su nam kasnije olakšali implementaciju same aplikacije. Uz dokumentaciju napravili smo i predložak UI dizajna koji je kasnije uvelike pomogao određivanju rasporeda elemenata na korisničkom sučelju i rješavanju nedoumica.
		 
		 U drugoj fazi projekta naglasak je bio na implementaciji aplikacije što je zahtjevalo više samostalnog rada od članova tima. Većina nas se prvi puta susrela sa implementacijom ovakvog projekta što je zahtjevalo dosta samostalnog učenja o korištenim tehnologijama kako bi ostvarili zadane ciljeve. Osim realizacije programskog rješenja u drugoj fazi bilo je potrebno dovršiti dokumentaciju i preostale UML dijagrame kako bi budući korisnici mogli lakše koristiti i vršiti preinake u sustavu. Također bilo je potrebno provesti testiranje. Zbog nekih grešaka u prvom dijelu izrade dokumentacije bilo je potrebno uvesti neke preinake, što je malo usporilo razvoj sustava, ali i dovelo do situacije da nismo implementirali neke funkcionalnosti:
		 \begin{packed_enum}
		 	
		 	\item[-] povezivanje s imenikom liječnika, trenutno imamo samo dodatnu tablicu u bazi koja se ponaša kao imenik liječnika
		 	\item[-] povezivanje s nacionalnim sustavom zdravstvene zaštite, MBO se trenutno provjerava iz dodane tablice u bazi
		 	\item[-] provjera vremena termina i terapija, bilo bi dobro da ima još provjera
		 	\item[-] nije implementiran unos lozinke kod promjene ovlasti djelatnika
		 \end{packed_enum}
		 
		 
		 Redovitim sastancima i komunikacijom preko WhatsAppa i Discorda postigli smo informiranost svih članova grupe o napretku projekta, ali i rješavanje različitih problema koji su se stvorili prilikom implementacije.
		 
		 Trenutno stanje projekta ima podosta mjesta za napredak i dodavanje različitih funkcionalnosti koje bi poboljšale iskustvo korisnika aplikacije, kao i moguća implementacija aplikacije za mobilne uređaje.
		 
		 Sudjelovanje na ovom projektu svim je članovima tima bilo vrijedno iskustvo, ne samo zbog pozitivnih dijelova kao što je usvajanje znanja izrade web aplikacija i dokumentacije, već i onih negativnih, mnogo smo naučili iz pogrešaka koje smo napravili putem, pogotovo što se tiče vremenske organizacije projekta. Zadovoljni smo postignutim bez obzira na puno prostora za napredak.
		 
		 
		 
		 
		  
		 
		 
		
		\eject 